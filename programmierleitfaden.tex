% documentation programmierleitfaden.tex
%
% a style guide / for naming conventions / collection of tips and tricks 
% for programming at FVA-BW.
%
% Author: Dominik Cullmann, <dominik.cullmann@forst.bwl.de>
% Version: $Id$

\documentclass[twoside]{scrreprt}
\usepackage[utf8]{inputenc}
\usepackage[T1]{fontenc}
\usepackage{lmodern}
\usepackage{adctex}
\usepackage{makeidx}
\usepackage[notlot,notlof]{tocbibind}
\usepackage{timestamp}
\usepackage{svn} % Ich arbeite mit GIT, aber svn macht einfache ID.
\usepackage{fancyhdr}
\usepackage{enumitem} % fuer [resume]
\usepackage{lastpage}
\usepackage[ngerman, english]{babel}
\usepackage[usenames,dvipsnames]{color}
\usepackage{listings}
\usepackage{verbatim}
\SVN $Id$
\fancyfoot{}
\fancyhead{}
\fancyhead[LE,RO]{git ID: \SVNId}
\fancyfoot[LO,RE]{\jobname{}.tex, \TeX{}ed \timestamp}
\fancyfoot[RO,LE]{\thepage{} of \pageref{LastPage}}
\pagestyle{fancy}
\providecommand{\refitem}[1]{\ref{item:#1}}
\providecommand{\labitem}[1]{\label{item:#1}}
\providecommand{\R}{\texttt{R}}
\providecommand{\code}[1]{\texttt{#1}}

\definecolor{mygreen}{rgb}{0.0,0.4,0.2}
\definecolor{mygray}{gray}{0.3}
\definecolor{mymauve}{RGB}{52,16,67}

\usepackage[
colorlinks=true,
linkcolor=mymauve,
urlcolor=mygreen,
citecolor=mygray,
plainpages=false
]{hyperref}


\lstset{ %
    inputpath=listings,
  %% backgroundcolor=\color{white},
  basicstyle=\footnotesize\ttfamily,
  %% identifierstyle=\color{mygray},%\sffamily,
  %% breakatwhitespace=false,
  %% breaklines=true,
  caption=\lstname,
  %% captionpos=b,
  commentstyle=\color{PineGreen},%\sffamily\bfseries,
  %% deletekeywords={...},
  %% escapeinside={\%*}{*)},
  %% extendedchars=true,
  %% frame=single,
  %% keepspaces=true,
  keywordstyle=\color{blue},%\rmfamily,
  language=R,
  %% morekeywords={*,...},
  %% numbers=left,
  %% numbersep=5pt,
  numberstyle=\tiny\color{mygray},
  %% rulecolor=\color{black},
  %% showspaces=false,
  showstringspaces=false,
  %% showtabs=false,
  %% stepnumber=2,
  stringstyle=\color{Sepia},%\rmfamily\itshape,
  %% tabsize=2,
  %% title=\lstname
  linewidth=\textwidth
}
\def\mystyle{}


% http://mintaka.sdsu.edu/GF/bibliog/latex/floats.html
% Alter some LaTeX defaults for better treatment of figures:
    % See p.105 of "TeX Unbound" for suggested values.
    % See pp. 199-200 of Lamport's "LaTeX" book for details.
    %   General parameters, for ALL pages:
\renewcommand{\topfraction}{0.9}	% max fraction of floats at top
\renewcommand{\bottomfraction}{0.8}	% max fraction of floats at bottom
    %   Parameters for TEXT pages (not float pages):
\setcounter{topnumber}{3}
\setcounter{bottomnumber}{2}
\setcounter{totalnumber}{4}     % 2 may work better
\setcounter{dbltopnumber}{2}    % for 2-column pages
\renewcommand{\dbltopfraction}{0.9}	% fit big float above 2-col. text
\renewcommand{\textfraction}{0.07}	% allow minimal text w. figs
    %   Parameters for FLOAT pages (not text pages):
\renewcommand{\floatpagefraction}{0.7}	% require fuller float pages
	% N.B.: floatpagefraction MUST be less than topfraction !!
\renewcommand{\dblfloatpagefraction}{0.7}	% require fuller float pages
	% remember to use [htp] or [htpb] for placement

\providecommand{\proot}{%
H:\textbackslash{}FVA-Projekte\textbackslash{}Methodenforum\textbackslash{}%
}
\providecommand{\pdir}{\proot{}programmierleitfaden\textbackslash{}}
\begin{document}


\selectlanguage{ngerman}
\title{Programmierleitfaden an der FVA Baden--W\"u{}rttemberg\\
\vspace{1cm}
\normalsize{Version \input{VERSION}}
}
\author{Arno R\"oder, Bernhard B\"o{}sch, Christian Vonderach,\\
Dominik Cullmann, Gerald K\"a{}ndler, Marco Reimann}
\uppertitleback{%
    Wir haben diesen Leitfaden f\"u{}r den Hausgebrauch an der FVA geschrieben,
    Pfadangaben wie \pdir{} ergeben au\ss{}erhalb der FVA keinen Sinn. 
    Falls Du diesen Leitfaden anderenorts nutzen oder weitergeben m\"o{}chtest, 
    solltest Du die Archivdatei \newline
    \pdir{}programmierleitfaden.zip \newline
    kopieren, sie enth\"a{}lt diesen Leitfaden und alle zugeh\"o{}rigen 
    Beispiele.

    \medskip
Kommentare, Fragen, Kritik und Korrekturvorschl\"a{}ge kannst Du gerne an
uns richten: <dominik.cullmann@forst.bwl.de> 
}
\lowertitleback{%
Copyright:\newline
© Forstliche Versuchs- und Forschungsanstalt Baden-W\"u{}rttemberg\newline 
Wonnhaldestra\ss{}e 4, 79100 Freiburg, \url{www.fva-bw.de}

\smallskip
This work is licensed under the 
Creative Commons Attribution 4.0 International License.
To view a copy of this license, 
visit \url{http://creativecommons.org/licenses/by/4.0/}.



}

\maketitle
\tableofcontents
\thispagestyle{fancy}
\chapter{Worum geht's?}
Mit diesem Leitfaden m\"o{}chten wir Dir, liebe Leserin, lieber Leser (und auch
uns selbst) helfen, weniger schlecht zu programmieren.
Es gibt an der FVA recht viele Menschen, die programmieren und nur wenige, die
dies richtig gelernt haben. Wenn wir uns auf einen gemeinsamen Satz von Regeln
einlassen k\"o{}nnen, werden wir den Code unserer Kolleginnen und Kollegen
leichter lesen und verstehen k\"o{}nnen.

Nun stellst Du Dir vielleicht die Frage: "`Wieso sollte jemand meinen Code lesen
wollen oder d\"u{}rfen?"'. Vielleicht wird ein anderer Mensch Dein Projekt
\"u{}bernehmen m\"u{}ssen, vielleicht bist Du sogar mutig genug, Deinen Code
anderen zu zeigen um zu sehen, was sie davon halten. Ziemlich sicher aber wirst
Du selbst Deinen Code wieder fr\"u{}her oder sp\"a{}ter wieder lesen und 
verstehen m\"u{}ssen. Und Du wirst Dir selbst sehr dankbar sein, wenn Du heute 
an Dein zuk\"u{}nftiges Ich denkst und Dich an diesem Leitfaden orientierst.

Manche Regel wird Dir eventuell zun\"a{}chst kleinlich erscheinen,
beispielsweise die mit den Leerzeichen  um  Operatoren (siehe Abschnitt
\ref{sec:format}), aber wenn Du \"o{}fter fremden oder eigenen etwas
angegammelten Code liest, wirst
du merken dasser sich fil laichter lesen laest falls sich der autorin an
bestimte regeln gehalten hat auch wen diese nicht einleuchteten ihm das
sich verhelt enlich wie mitt krammmatik unt rechtschreipunk

Da unseres Wissens derzeit weitaus die meisten programmierenden Mitarbeiterinnen
und Mitarbeiter an der FVA mit der Programmiersprache \R{} arbeiten, sind die
meisten Beispiele dieses Leitfadens in \R{} verfasst und wir haben beim
Zusammenstellen der Regeln haupts\"a{}chlich an \R{} gedacht.
Sinngem\"a{}\ss{} lassen sich aber die meisten Regeln auf die meisten
anderen Sprache \"u{}bertragen. Wir haben sie ja auch nicht selbst erfunden
sondern nur zusammengetragen, gr\"o{}\ss{}tenteils aus \cite{cc} und \cite{wsp}.

Die Kapitel~\ref{sec:R1} und \ref{sec:R2} stellen die wichtigsten beiden Regeln
vor, Kapitel~\ref{sec:conventions} enth\"a{}lt alle Regeln, an die Du Dich
halten musst\footnote{
  Zur Verbindlichkeit unserer Regeln lies Abschnitt~\ref{sec:zwang}
}, und Kapitel~\ref{sec:tipps} schlie\ss{}t mit vielen sinnvollen Tipps.

S\"a{}mtliche Listings zeigen einen Dateinamen, diese und andere  Dateien 
findest Du im Verzeichnis  
\pdir.


\section{Die wichtigste Regel\label{sec:R1}}
Schreibe Deinen Code so, dass Du ihn in drei Monaten noch verstehen wirst.

\section{Die zweitwichtigste Regel\label{sec:R2}}
Schreibe Deinen Code so, dass Du ihn in sechs Monaten noch verstehen wirst. 
Nein, Unfug, richtig ist: schreibe Code f\"u{}r Menschen und nicht f\"u{}r 
Maschinen. Die meisten Interpreter und Compiler sind heute so gut, dass es
kaum lohnt, Code zu optimieren. Also lass es. Es sei denn, Du wei\ss{}t
genau, was Du tust. 

Versuche nicht, Deine Leser zu beeindrucken, versuche, Ihnen verst\"a{}ndlich zu
machen, was Du tust \ldots oder eigentlich h\"a{}ttest tun wollen.
Die Programme in  Listing~\ref{lst:uncool} und Listing~\ref{lst:cool} geben
jeweils einen String aus. Lies beide und
\"u{}berlege, welcher String wohl die jeweilige Ausgabe ist. Vergleiche dann die
Zeiten, die Du f\"u{}r Deine \"U{}berlegungen ben\"o{}tigt 
hast. 
\lstinputlisting[float=h, label=lst:uncool, language=C]{uncool.c}

\lstinputlisting[float=h, label=lst:cool, language=C]{cool.c}

Wenn Du willst, kannst Du  beide Programme komplieren und ihre Ausgaben mit
Deinen \"U{}berlegungen vergleichen\footnote{%
\selectlanguage{english}%
Oder Du glaubst uns: uncool.c gibt 
"\catcode`_=12 \ttfamily \input{|"gcc uncool.c -o uncool && ./uncool"}"
aus und cool.c 
"\catcode`_=12 \ttfamily \input{|"gcc cool.c -o cool && ./cool"}".
\selectlanguage{ngerman}
}%
. 

\selectlanguage{ngerman}
\chapter{Konventionen\label{sec:conventions}}
\section{Verbindlichkeit der Konventionen\label{sec:zwang}}
Alle unsere Regeln sind unverbindlich. Aber wenn Du bis hierher gelesen hast,
hoffen wir, dass Du dennoch entschieden hast, Dich an sie halten zu wollen.
Alternativ kannst Du auch einfach \cite{KR} lesen und alles so machen wir 
Kernighan und Ritchie.

\section{Englisch oder nicht?}
\minisec{Wider}
\begin{itemize}
\item Englisch ist wahrscheinlich nicht Deine Muttersprache.
\end{itemize}
\minisec{F\"ur}
\begin{itemize}
\item Gehe nicht davon aus, dass Leserinnen Deines Codes Deine
  Muttersprache teilen. Auch nicht an der FVA.
\item Alle Programmiersprachen, die wir nutzen, und
  \href{http://en.wikipedia.org/wiki/Non-English-based_programming_languages}
  {weitaus die meisten}
  aller existierenden Programmiersprachen sind in Englisch implementiert.
  Gehe also davon aus, dass die meisten Menschen, die Deine Codes lesen
  k\"o{}nnten, zumindest rudiment\"a{}re Englischkenntnisse besitzen.
\item Stell Dir vor, Du \"u{}bernimmst ein Projekt. M\"o{}chtest Du Code wie 
    in Listing \ref{lst:meghal} 
    lesen\footnote{Dieses Beispiel stammt aus \cite{wsp}.}?
\end{itemize}
\begin{lstlisting}[float=h, label=lst:meghal, language=Java]
/*
 * Kincskereso halalat vezenzli le.
 */

public void meghal()
{
  Game.jatekVege();
  this.mezo.setCellaElem(null);
}
\end{lstlisting}

\minisec{Welches Englisch?} Britisches Englisch mag \"a{}sthetischer sein,
aber da die meiste Software in amerikanischem Englisch geschrieben wird: halte
es ebenso. Warum? Weil dann Menschen, die Deinen Code benuzten 
(auch Du selbst!), vorhersagen k\"o{}nnen, dass die Variable
\code{background\_colors} nicht \code{background\_colours} hei\ss{}t (mehr zu
Namen gibt's in Abschnitt~\ref{sec:naming}). 

\minisec{Kompromiss} Ja, die Umgangssprache an der FVA ist Deutsch, und Du
kannst davon ausgehen, dass die meisten Leser an der FVA besser Deutsch als
Englisch lesen. Wenn Du glaubst, dass alle Deine Leserinnen nur an der FVA
sein werden und es Dir \emph{wirklich} schwer f\"a{}llt, Englisch zu schreiben
und Du auch nicht mit dem Programmieren nebenbei Dein Englisch ein wenig
verbessern willst: dann schreib' auf Deutsch.

\section{Kommentare}
\subsection{Verst\"a{}ndniskommentare\label{sec:compcom}}
Es gibt Menschen, die Kommentare im Code f\"u{}r sch\"a{}dlich halten -- weil 
Code so verst\"a{}ndlich sein sollte, da\ss{} er unkommentiert verstanden
werden kann.
Und weil die Kommentare oft veralten, w\"a{}hrend Du den Code \"a{}nderst.
Diese Menschen sind sehr gute Programmierer.
Da Du das wahrscheinlich nicht bist:
Kommentiere viel. Und aktualisiere die Kommentare! 
Mittelschlecher Code mit Kommentaren liest sich besser als mittelschlechter  
Code ohne Kommentare. Es sei denn, die Kommentare sind veraltet und/oder 
widersprechen dem Code.

Kommentiere das \emph{Was}, nicht das \emph{Wie}, das \emph{Wie} steht schon im 
Code, und der hat immer recht.
Und da sich das Wie meist schneller \"a{}ndert als das Was, musst Du auch
nicht so viele Kommentare aktualisieren.
Falls es ein \emph{Warum} gibt: schreib es auf. 
Gut ist  Listing \ref{lst:comment}.
\lstinputlisting[float=h, label=lst:comment, firstline=2]{sum_up.r}

Nutze Schl\"u{}sselworte in Deinen Kommentaren. Halte Dich dabei an die folgende
Liste:
\begin{description}
\item[TODO:] Hier gibt es was zu verbessern. Aber der Code tut, was er soll.
  Vielleicht ist er h\"a{}sslich, unverst\"a{}ndlich oder langsam,
  aber er ist nicht gef\"a{}hrlich.
\item[FIXME:] Hier ist etwas kaputt und sollte sofort repariert werden.
\item[XXX:] Hier ist etwas v\"o{}llig kaputt und sollte schon repariert
  worden sein.
\end{description}
Und finde heraus, ob Dein bevorzugter Editor oder Deine IDE (integrated
development environment) diese Schl\"u{}sselworte erkennt oder wie Du sie
ihm/ihr beibringst.

\subsection{Dokumentationskommentare\label{sec:doccom}}
Jedem Modul oder Paket, jeder Methode, Klasse oder Funktion steht ein
Dokumentationskommentar zu Gesicht. Dokumentationskommentare haben mindestens 
folgende Struktur:
\begin{enumerate}
    \item Eine Titelzeile,
    \item Eine (mehrzeilige) Beschreibung,
    \item Eine Autorenangabe (am besten mit einer Kontaktinformation) und
    \item Eine Versionsangabe (im einfachsten Fall das \"A{}nderungsdatum).
\end{enumerate}
Die Titelzeile ist immer die erste Zeile eines Dokumenationskommentares, sie
besteht aus einem Satz ohne schlie\ss{}endes Satzzeichen und ist durch eine
Dokumentationskommentarleerzeile von der Beschreibung getrennt. Diese ist
wiederum durch eine Dokumentationskommentarleerzeile von den \"u{}brigen Angaben
getrennt%
\footnote{Diese Struktur erlaubt es, aus Dokumentationskommentaren
automatisiert e-Mails zu erstellen. Halte sie bitte ein.}%
, die mit einem "`\code{Schl\"u{}sselwort:}"' beginnen.

F\"u{}r eine Funktion sollten auch 
\begin{enumerate}[resume]
    \item (kein bis f\"u{}nf)\footnote{siehe
	    Abschnitt~\ref{sec:parameteranzahl}} Parameter (wobei das erste Wort
	    nach "`\code{Parameter:}"' den Namen des Parameters angibt, der 
	    folgende Rest die Beschreibung des Parameters liefert) und 
    \item eine R\"u{}ckgabe
\end{enumerate}
dokumentiert werden. Weitere zus\"a{}tzliche Schl\"u{}sselworte wie 
\emph{Copyright}, \emph{Note}, \emph{Warning} oder \emph{Known Bugs} 
schaden nicht.
In Listing~\ref{lst:doccom1} siehst Du einen Dokumentationskommentar
f\"u{}r eine kleine Funktion (die sehen wir sp\"a{}ter in
Listing~\ref{lst:write_readme}).
\lstinputlisting[float=h, label=lst:doccom1, firstline = 2]{write_readme_doc.r}

\subsection{\R{}-Auswertungsskripte dokumentieren
\label{sec:rdoc}
\index{R!Auswertungsskripte!dokumentieren|textbf}}
Viele von uns programmieren in der Sprache \R{}, und vielleicht schreibst 
auch Du nicht so viele Funktionen, sondern eher Auswertungsskripte f\"u{}r 
wissenschaftliche Fragestellungen. 
Auch solche Programme solltest Du kommentieren, und zwar sowohl mit 
Verst\"a{}ndnis- als auch gegebenenfalls mit  Dokumentationskommentaren, wie wir
sie gerade in den Abschnitten \ref{sec:compcom} und \ref{sec:doccom} gesehen 
haben.

Aber der wichtigste Kommentar eines \R{}-Auswertungsskriptes steht  
\index{R!Auswertungsskripte!dokumentieren|textbf}
am Anfang der Datei: dieser Dokumentationskommentar bezieht sich nicht auf eine
einzelne Definition (einer Funktion, Klasse oder Methode), sondern auf den 
gesamten Inhalt der
Datei. Er wird daher oft als "`(Datei-)Header"' bezeichnet und tr\"a{}gt
h\"a{}ufig den Dateinamen in der Titelzeile.
Listing~\ref{lst:rdoc} zeigt einen solchen Dokumentationskommentar. 
\index{R!Auswertungsskripte!dokumentieren|textbf}
\lstinputlisting[float=h, label=lst:rdoc, firstline = 2]{header.r}
 
In Abschnitt~\ref{sec:documentation} werden wir sehen, wie wir solche
Kommentare automatisiert verarbeiteten k\"o{}nnen.


\section{Namesgebung\label{sec:naming}}
\subsection{Worttrennung}
Es gibt viele m\"o{}gliche Konventionen zur Gro\ss{}- und Kleinschreibung  von
Namen und
Trennung von W\"o{}rten in Namen:
\begin{itemize}
\item UpperCamelCase
\item lowerCamelCase
\item lowercase
\item UPPERCASE
\item lower\_case\_with\_underscores,
\item UPPER\_CASE\_WITH\_UNDERSCORES
\item Upper\_Camel\_Case\_With\_Underscores
\end{itemize}
Wenn Du in ein Projekt einsteigst, halte Dich am besten an die (hoffentlich)
vorhandene Konvention.
Wenn Du ein neues Projekt beginnst: Sieh' nach, ob es f\"u{}r die von Dir 
verwendete Sprache eine offizielle Namenskonvention gibt. Suche im Internet 
beispielsweise nach "`Java naming conventions"'.

\paragraph{Die Sprache \R{}\index{R!Namenskonvention}} besitzt keine offizielle
Namenskonvention. Derzeit
popul\"a{}re "`style guides"', die auch Namenskonventionen enthalten, sind die
von
\href{https://google-styleguide.googlecode.com/svn/trunk/Rguide.xml}{google} und
\href{http://adv-r.had.co.nz/Style.html}{Hadley Wickham}.
Wir empfehlen f\"u{}r neue Projekte in \R{}: 
ExceptionNames und ClassNames, GLOBAL\_VARIABLES, KONSTANTS und sonst (also
f\"u{}r Methoden, Funktionen, Variablen, Module, Instanzvariablen und
Funktionsparameter) nur 
lower\_case\_with\_underscores\index{R!Namenskonvention}.
Das entspricht dem derzeitigen Stand der
\href{http://legacy.python.org/dev/peps/pep-0008/#naming-conventions}{python
  naming conventions}, die Namenskonvention Hadley Wickhams ist eine
  Untermenge unserer Empfehlung.

\minisec{R packages\index{R!Paketnamen} } d\"urfen keine Unterstriche im Namen 
tragen, wir
empfehlen daher f\"u{}r \emph{\R{} packages} abweichend lowerCamelCase, 
da Listing~\ref{lst:rpackages} derzeit ein Ergebnis gr\"o{}\ss{}er eins 
liefert (es also mehr cranPackages als CranPackages gibt).
\lstinputlisting[float=h, label=lst:rpackages, firstline = 3]%
{query_package_names.r}

\subsection{Sprechende Namen}
Die gute Bennenung von Bezeichnern (Variablen, Funktionen, Klassen, \ldots)
ist in etwa
das Schwierigste am Programmieren. Gute Name sprechen. Sie sind gut leserlich,
nicht leicht zu verwechseln und logisch (also nicht: \code{l1I},
\code{international\_f\emph{or}m} und \code{international\_f\emph{ro}m} im 
selben Code -- auch wenn das mit einem guten Editor oder einer IDE nur halb so 
schlimm ist -- oder \code{left\_column} statt \code{navigation\_column}: die 
\emph{Lage} der Spalte ist weder relevant noch zwingend fix, ihre 
\emph{Funktion} hingegen beschreibt sie gut).
Sie sind nicht allzu lang und vor allem nie zu kurz.
\code{val}, \code{ding}, \code{foobar} gehen gar nicht%
\footnote{Einzig \code{x, y, z} f\"ur Koordinaten
  und \code{i, j, k}\ldots f\"ur tempor\"a{}re Z\"a{}hler in Schleifen sind
  akzeptabel.}.
Am besten folgen sie einem klaren Schema und sind dadurch erratbar.

\subsection{Abk\"u{}rzungen}
\begin{enumerate}
\item\labitem{abbrevs} Namen sollten \emph{niemals} abk\"u{}rzen.
\item \emph{Einzige} Ausnahme von Regel \refitem{abbrevs} bilden
  Abk\"u{}rzungen, die allgemein bekannter sind als ihre ausgeschriebene
  Variante\footnote{"`allgemein"' ist schwammig: Bei Menschen, die Deine
    Programme lesen, d\"u{}rfte allgemein URL bekannter sein als uniform
    resource locator. Aber bei SPD werden sie, je nach dem ob sie sich
    sonst eher mit Graphikdesign,  Wahrnehmungsst\"o{}rungen, Radeln
    oder Politik herumplagen, an die Society of Publication Designers,
    Sensory Processing Disorder, Shimano Pedaling Dynamics oder die
    Sozialdemokratische Partei Deutschlands denken. Gehe vor allem nicht
    davon aus, dass sie mit Deinem Kulturkreis vertraut sind.}.
\item Die einzige \emph{andere} Ausnahme von Regel \refitem{abbrevs} bilden
  Abk\"u{}rzungen in Tabelle~\ref{tab:abbrevs}.
\end{enumerate}
\begin{table}[h!]
  \centering
  \begin{tabular}{l|l}
    Abk\"u{}rzung & Name \\ \hline
    len & length\\
    max & maximum\\
    min & minimum\\
    num & number\_of \\
    pos & position \\
    tmp & temporary\\
    val & value
  \end{tabular}
  \caption{Die einzigen \emph{anderen} Abk\"u{}rzungen.}
  \label{tab:abbrevs}
\end{table}

\subsection{Boolsche Variable}
Boolschen Variablen, also solchen, die entweder wahr oder falsch sind, solltest
Du zur besseren Verst\"a{}ndlichkeit und Abgrenzung ein einleitendes \code{is\_}
spendieren: \code{ if~(is\_defined)~\{\ldots\}} ist klarer als
\code{if~(defined)\{\ldots\}}.

\subsection{Ein Name f\"u{}r viele Dinge}
Wenn ein Name etwas bezeichnet, das aus vielen Elementen besteht (eine Liste,
einen Array oder Vektor), so gibt ihm ein Plural-S:
\begin{lstlisting}
  background_color <- 'blue'
  foreground_colors <- c('red', 'black')
\end{lstlisting}

\subsection{Funktionsnamen}
Funktionen tun etwas, spendiere Ihnen daher ein \emph{einleitendes} Verb:
\code{show\_help()} statt \code{help()} (und schon gar nicht  
\code{help\_show()}).

\section{Formatierung\label{sec:format}}
Die Formatierung eines Quellcodes ist in etwa \"a{}quivalent zur
Rechtschreibung. Wir raten Dir daher:
\begin{itemize}
\item folge dem "` one true brace style"' (%
\href{http://en.wikipedia.org/wiki/Indent_style#Variant:_1TBS}{1TBS}).\\
  In erster Linie bedeutet dies: Bl\"o{}cke werden immer geklammert,
  \"o{}ffnende \{ kommen ans Ende einer
      Zeile\footnote{Au\ss{}er bei Funktionsdefinitionen pr\"a{}prozessierter
      Sprachen, also etwa \code{C}, hier steht sie in der ersten Spalte einer
  neuen Zeile.}, schlie\ss{}ende \} an einen Zeilenanfang, 
      vor einer
  \"o{}ffnenden \{ steht ein Leerzeichen, vor $($ ebenso, es seid denn,
  der voranstehende Bezeichner ist ein Funktions- oder Methodenname.
\item zu einer Zeilenbreite von h\"o{}chstens 80 Zeichen. Das passt
  noch auf eine gedruckte A4-Seite. Wenn Du mehr brauchst sind
  entweder Deine Namen zu lang oder Deine Verschachtelungen zu
  tief (siehe Abschnitt \ref{sec:breite}).
  Bringe Deinem Editor oder Deiner IDE bei, bei 80 Zeichen hart
  umzubrechen.
\item zu einer Einr\"u{}ckungstiefe 4 Leerzeichen, falls Deine
  Programmiersprache nichts anderes \emph{erzwingt}.
  (Die Einr\"u{}ckungstiefe ist eigentlich v\"o{}llig egal, Du
  musst Dich nur festlegen. Und Leerzeichen statt
  Tabulatoren verwenden.)
\item fasse Operatoren mit Leerzeichen ein.
\item lasse jedem Komma ein Leerzeichen folgen.
\item schreibe mehrteilige Bedingungen auf mehrere Zeilen.
\item breche mehrzeilige Anweisungen nach Operatoren oder Kommata um.
\item r\"u{}cke mehrzeilige geklammerte Bl\"o{}cke (etwa Funktionsdefinitionen 
    mit mehrzeiliger Parameterliste oder mehrteilige Bedingungen) so ein, dass 
    sich die Einr\"u{}ckung der Folgezeilen in diesem Block an der Position der
    \"o{}ffnenden Klammer orientiert (jeder gute Editor, jede gute IDE macht das
    automatisch f\"u{}r Dich).
\end{itemize}
Ein Beispiel der obigen Empfehlungen 
findest Du in Listing~\ref{lst:format}.
\lstinputlisting[float=h, label=lst:format, firstline = 2]{format.r}


\chapter{Tipps\label{sec:tipps}}
\section{Stilfragen}
\subsection{Dateien}
Fasse zusammengeh\"o{}rige Funktionen, Klassen und so fort in Dateien zusammen; 
trenne Dateien auf, wo es sich anbietet.
Eine Datei sollte nicht zu lang sein. F\"u{}nf Bildschirmseiten sind
lang (bei einer Aufl\"o{}sung von 1280 x 1024 Pixel, einer
Schriftgr\"o{}\ss{}e von 10 Punkt und ein wenig Platzverbrauch durch die
Bedingunselemente Deines Editor/Deiner IDE entspricht das etwa 300
Zeilen).

%diese info noch irgendwo unterbringen, oder hab ich sie verpasst?
\subsection{Funktionen}
Wenn Du merkst, dass Du eine Code-Teil per copy-and-paste vervielf\"a{}ltigst
(und dann eventuell leicht, immer an den selben Stellen, modifizierst),
solltest Du ihn in einer Funktion zusammenfassen (die modifizierten werden die
Funktionsparameter). 
\"A{}nderungen musst Du dadurch nur an einer Stelle vornehmmen, Fehler 
findest Du leichter.

\subsection{Funktionsl\"a{}nge} 
Ein Dutzend Zeilen sind gut, 50 sind zu lang.
Lagere einzelne Teile aus (siehe Abschnitt~\ref{sec:if}).

\subsection{Funktionsbreite\label{sec:breite}} 
Verschachtele nicht zu tief. Drei Ebenen sollten reichen. 
Lagere einzelne Zweige in eigene Funktionen aus.

\subsection{Kurze Einr\"u{}ckungen} 
Einr\"u{}ckungen, zum Beispiel bei Schleifen oder \code{if}-Zweigen,
die sich \"u{}ber viele (beispielsweise mehr als 12 Zeilen) erstrecken, 
verlangen nach einer eigenen Funktion.

\subsection{if-Verschachtelungen\label{sec:if}}
In Listing~\ref{lst:if1} siehst Du eine \code{if}-Verschachtelung, leider ist
der Code recht breit, er hat immerhin zwei statt besser eine Verschachtelung.
Au\ss{}erdem wird erst auf den R\"u{}ckgabewert von
\code{file.exists()} getestet, aber der Fehlerfall f\"u{}r
\code{exists()} zuerst bearbeitet.
\lstinputlisting[float=h, label=lst:if1, firstline = 3]{if_1.r}
Das macht das Ganze zwar nicht v\"o{}llig unverst\"a{}ndlich, aber
Listing~\ref{lst:if2} ist deutlich lesbarer:
\lstinputlisting[float=h, label=lst:if2, firstline = 4]{if_2.r}
Das Schreiben des README ist in
einer eigenen Funktion verschwunden, deren Name verst\"a{}ndlich ist. Wenn ich
mich f\"u{}r die Details dieser Funktion 
(die wir in Listing~\ref{lst:write_readme} sehen und die in
Listings~\ref{lst:doccom1} beziehungsweise \ref{lst:doccom2} dokumentiert ist;
die Funktion samt Dokumentationskommentar findest Du in 
Listing~\ref{lst:write_readme_doccom} 
im Anhang \ref{app:R})
nicht interessiere, ist der Rest des Codes viel besser verst\"a{}ndlich.
\lstinputlisting[float=h, label=lst:write_readme, firstline=12
]{write_readme.r}

Falls Du tiefe \code{if-else}-B\"a{}ume schreibst, brauchst Du eventuell 
ein \code{case} (in \R{} hei\ss{}t es \code{switch}).
\subsection{Parameteranzahl\label{sec:parameteranzahl}} 
Funktionen sollten nicht mehr als 5 Parameter besitzen.
Brauchst Du mehr, solltest Du Teile der Funktion in Hilfsfunktionen auslagern
oder zusammengeh\"o{}rige Parameter in Arrays \"u{}bergeben.

\subsection{Fremder Code} 
Wenn Du fremden Code nutzen willst, solltest Du ihn in separate Dateien kopieren
und in Deinen Code einbinden. 
Formatiere ihn nicht entsprechend Deinen Konventionen um, 
kommentiere ihn nicht au\ss{}er durch eine Quellenangabe. 
Und achte auf das Copyright (siehe Abschnitt~\ref{sec:copy_foreign}).

\section{Entwicklungsumgebungen}
Es gibt viele Editoren und Entwicklungsumgebungen, falls Du neu an der FVA bist,
noch keine
eigenen Pr\"a{}ferenzen hast und in der Sprache \R{} schreibst, empfehlen wir 
Dir die Software \href{http://www.rstudio.com/products/rstudio/}{RStudio}, 
\index{R!Entwicklungsumgebung} 
ohne Support ist sie unter der AGPL v3 ver\"o{}ffentlicht.

\section{Versionskontrollsysteme\label{sec:vcs}}
Programmieren ohne Versionskontrollsystem ist wie Computerspielen ohne 
Spielstandspeicherung: ein Fehler, und alles ist futsch.
Tu Dir den Gefallen, und nutze eines.
Verlasse Dich nicht auf normale Backups, stricke Dir kein eigenes System selbst,
nutze eines, das die Profis freundlicherweise schon getestet haben. 
An der FVA haben wir gute Erfahrungen mit 
subversion (\url{http://subversion.apache.org/})
und git (\url{http://git-scm.com/}).
Wenn Du Hilfe brauchst: frage Arno R\"o{}der oder Dominik Cullmann.  

\section{Dokumentationssyteme}
Die meisten Sprachen besitzen Dokumentationssyteme, die Dokumentationskommentare
auslesen und Dokumentationen daraus fertigen (etwa Doxygen, das C, C++, Java und
Python erkennt; f\"u{}r die letzten beiden Sprachen gibt es aber auch eigene
Systeme: JavaDoc und docstrings). 
Nebenbei l\"a{}sst sich meist automatisch testen, ob der
Kommentar mit dem beschriebenen Objekt \"u{}bereinstimmt. 

\section{Encoding}
Es gibt nicht "`einfach nur Text"'. "`Einfach nur Text"' bekommt man mit Papier 
und Bleistift. "`Einfacher Text auf dem Bildschirm"'ist wahrscheinlich
ASCII. ASCII stammt aber aus einer Zeit, in der Computernutzer sich auf das
lateinische Alphabet beschr\"a{}nken mussten. Schon deutsche Umlaute sind nicht
mehr im einfachen (jedoch im erweiterten) ASCII-Zeichensatz verf\"u{}gbar.
Griechische, arabische oder kyrillische Buchstaben schon gar nicht. 


Falls Du Programme schreibst, die als Eingabe Zeichen wie ©  oder ç haben
\emph{k\"o{}nnten}, solltest Du dich mit Encodings auseinandersetzen. Dabei 
ist wichtig, dass Du die Eingabe entsprechend kodierst und das Programm exakt
diese Kodierung erwartet. Die Datei, aus der dieser Leitfaden kompiliert
wird, ist beispielsweise 'utf-8'-kodiert, der Compiler, der sie in ein pdf 
setzt, erh\"a{}lt die Anweisung \verb+\inputencoding{utf8}+: sowohl 
tats\"a{}chliche als auch die erwartete Eingabe sind also 
\href{http://en.wikipedia.org/wiki/UTF-8}{UTF-8}, eine Umsetzung des
\href{http://en.wikipedia.org/wiki/Unicode}{Unicode}-Standards.
\section{Urheberrechte und Lizenzen}
\subsection{Eigener Code}
Ma\ss{}geblich ist das deutsche "`Gesetz \"u{}ber Urheberrecht und verwandte
Schutzrechte"' (UrhG). 
Urheber eines Programmes bist immer Du selbst, das Urheberrecht ist auch nicht
\"u{}bertragbar (\$ 29 UrhG).
Das UrhG enth\"a{}lt derzeit in Abschnitt 8 "`Besondere Bestimmungen
f\"u{}r 
Computerprogramme"' folgenden Paragraphen:
\minisec{\$ 69b Urheber in Arbeits- und Dienstverh\"a{}ltnissen}
(1) Wird ein Computerprogramm von einem Arbeitnehmer in Wahrnehmung seiner
Aufgaben oder nach den Anweisungen seines Arbeitgebers geschaffen, so ist
ausschlie\ss{}lich der Arbeitgeber zur Aus\"u{}bung aller 
verm\"o{}gensrechtlichen Befugnisse an dem Computerprogramm berechtigt, sofern 
nichts anderes vereinbart ist.\\*
(2) Absatz 1 ist auf Dienstverh\"a{}ltnisse entsprechend anzuwenden.

Das bedeutet, dass Du zwar Urheber des Codes bist, er Dir aber nicht
geh\"o{}rt. Bevor Du also etwa Code unter einer Lizenz Deiner Wahl
ver\"o{}ffentlichst, solltest Du Dir das Einverst\"a{}ndnis Deiner Vorgesetzten 
einholen.

\subsection{Fremder Code\label{sec:copy_foreign}}
Viel Code ist unter freien Lizenzen verf\"u{}gbar. Einige, etwa die
\href{https://creativecommons.org/licenses/by-sa/4.0/legalcode}{creative commons
Attribution-ShareAlike 4.0 International (cc-sa)} oder die
\href{http://www.gnu.org/licenses/gpl-3.0.html}{GNU General Public License
(GPL)} verlangen, dass Lizenznehmer (also: Du) ihre Programme nur dann
ver\"o{}ffentlichen, wenn sie den Code unter der selben Lizenz
ver\"o{}ffentlichen\footnote{Es gibt auch weniger strikte freie Lizenzen,
beispielsweise die 
\href{http://opensource.org/licenses/BSD-2-Clause}{BSD 2-Clause License} und
die 
\href{http://opensource.org/licenses/MIT}{MIT License}%
. }.
Da die meisten Progamme, die Du schreiben wirst weder bin\"a{}r noch als
Quellcode ver\"o{}ffentlicht zu werden drohen, kann Dir das eigentlich gleich 
sein.
Da die von Dir erstellten Programme aber Deinem Arbeitgeber geh\"o{}ren und der
sie bin\"a{}r oder als Quellcode ver\"o{}ffentlichen wollen  k\"o{}nnte, 
solltest Du, wenn Du fremden Code unter solchen Lizenzen in Deinen Programmen
nutzt, dies vermerken.
\section{\R{}}
\subsection{Anleitungen\index{R!Anleitungen} }
Die offiziellen  \R{}-Anleitungen findest Du lokal auf Deinem Rechner per 
\code{help.start()}, oder auf \url{http://cran.r-project.org/doc/manuals/}.
Besonders hilfreich ist 
\href{http://cran.r-project.org/doc/manuals/r-release/R-intro.pdf}
{"`An Introduction to R"'}.

Inoffizielle Anleitungen findest Du unter 
\url{http://cran.r-project.org/doc/contrib/}, ihre Beschreibung unter 
\url{http://cran.r-project.org/other-docs.html}. Hilfreich ist 
\href{http://cran.r-project.org/doc/contrib/Paradis-rdebuts_en.pdf}
{"`R for Beginners"'} 
und, auf deutsch, 
\href{http://cran.r-project.org/doc/contrib/Grosz+Peters-R-Reader.pdf}
{"`R Reader"'}.

\href{http://www.burns-stat.com/pages/Tutor/R_inferno.pdf}{The \R{} Inferno} ist
eine hilfreiche Sammlung klassischer Fallstricke in R. 

\subsection{Dokumentationssystem\index{R!Dokumentationssystem}}
F\"u{}r die Sprache \R{}
gibt es kein offizielles Dokumentationssytem, aber 
das Paket \href{http://cran.r-project.org/web/packages/roxygen2/index.html}
{\code{roxygen2}} versucht, dem abzuhelfen. 
Wenn wir unseren Dokumetationskommentar aus Listing~\ref{lst:doccom1} der 
Struktur von \code{roxygen2} anpassen, erhalten wir Listing~\ref{lst:doccom2}.

\lstinputlisting[float=h, label=lst:doccom2, firstline = 2, lastline=11]%
{write_readme.r}

Das ist inhaltsgleich zu  Listing~\ref{lst:doccom1}, nur die Version haben wir 
durch den ID-tag eines Versionskontrollsystems (siehe Abschnitt~\ref{sec:vcs}) 
ersetzt (damit wir das Datum nicht mehr von Hand
\"a{}ndern m\"u{}ssen, sondern unserer Versionskontrollsystem uns die Arbeit
abnimmt).
Aus Listing~\ref{lst:doccom2}  kann \code{roxygen2} eine Dokumentation in pdf 
erstellen, die Du unter \href{write_readme.pdf}{\pdir{}write\_readme.pdf} sehen 
kannst.

\subsection{\R{}-Auswertungsskripte dokumentieren
\index{R!Auswertungsskripte!dokumentieren}
\label{sec:documentation}}
\code{roxygen2} dient der Dokumentation von \R{}-Paketen, nicht von
Auswertungsskripten. Es kann daher mit Anweisungen, die nicht Teil von
Funktionen, Klassen und so weiter sind, nicht umgehen. Dazu geh\"oren
selbstverst\"a{}ndlich auch unsere Header aus Abschnitt~\ref{sec:rdoc}.
Es spricht aber nichts dagegen, sich an die Struktur von \code{roxygen2} zu
halten und Listing~\ref{lst:rdoc} durch Listing~\ref{lst:rroxy} zu ersetzen.
\lstinputlisting[float=h, label=lst:rroxy, firstline = 2, lastline = 8]
{header_roxygen.r}

Im Gegenteil, mit Listing~\ref{lst:package} kannst Du Dir ein \R{}-Paket
installieren, mit dem Du 
Dateiheader\index{R!Auswertungsskripte!Dokumentationskommentar} 
\index{R!Auswertungsskripte!Dateiheader|see {Dokumentationskommentar}}
f\"u{}r \R{}-Auswertungsskripte in pdf 
umwandeln
kannst.\lstinputlisting[float=h, label=lst:package, firstline = 2]
{install_documentation.r}
 Wenn Du jetzt den Kommentar in  Listing~\ref{lst:rroxy} zu
Listing~\ref{lst:rroxy_working} 
\lstinputlisting[float=h, label=lst:rroxy_working, firstline = 2]
{header_roxygen.r}
erweiterst, erh\"altst Du durch
Listing~\ref{lst:roxy_ex} 
\lstinputlisting[float=h, label=lst:roxy_ex, linerange={2-2,4-4}]
{documentation_examples.r} die Datei \\*
\href{%
header_roxygen.pdf}{%  
\pdir{}header\_roxygen.pdf}.  
%%%%\clearpage

\minisec{Sweave\index{Sweave|textbf}} Alternativ kannst Du f\"u{}r 
Auswertungsskripte auch
\href{http://www.stat.uni-muenchen.de/~leisch/Sweave/}{\code{Sweave}} einsetzen.
Ein Beispiel zeigt Listing~\ref{lst:sweave} im
Anhang~\ref{app:sweave}.
Auch diese Datei hat einen Header, der analog zu den obigen aufgebaut ist, nur
f\"u{}r \TeX{}.
Aus  Listing~\ref{lst:sweave} kann die Datei \\* 
\href{%
template.pdf}{%
\pdir{}template.pdf}  
generiert werden.

\subsection{Skriptvorlage\index{R!Auswertungsskripte}
\index{R!Auswertungsskripte!Vorlage|textbf} }
Eine Vorlage f\"ur ein \R{}-Skript findest Du in Listing~\ref{lst:roxy_template}
im Anhang~\ref{app:R}. 
Lies sie aufmerksam, passe sie an Deine Bed\"u{}rfnisse an und gew\"o{}hne Dich
daran, Deine Version dieser Vorlage f\"u{}r alle Deine \R{}-Auswertungsskripte
zu nutzen.
Wir haben sie durch Listing~\ref{lst:template} erstellt.
\lstinputlisting[float=h, label=lst:template, linerange={2-2,6-6}]
{documentation_examples.r}


\bibliography{biblio}
\bibliographystyle{unsrt}
\printindex
\appendix
\chapter{Sweave\label{app:sweave}\index{Sweave}}
\lstinputlisting[label=lst:sweave, language=TeX]{template.Rnw}
\pagebreak
\chapter{\R{}-Beispiele\label{app:R}\index{R!Beispiele}}
\lstinputlisting[label=lst:write_readme_doccom]{write_readme.r}
\index{R!Auswertungsskripte!Vorlage}%
\lstinputlisting[label=lst:roxy_template]{my_r_file.r}%
%\clearpage
%\includepdf[pages=-]{write_readme.pdf}
\end{document}
